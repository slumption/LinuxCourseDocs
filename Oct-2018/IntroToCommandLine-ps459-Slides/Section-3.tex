\part{Anatomy of a command}
\begin{frame}
\partpage
\end{frame}

\section{Commands}
\begin{frame}{Section 3: Commands}
\begin{itemize}
\item A command is an instruction given by a user telling a computer to do something
\item Commands often take options
\item Commands often take arguments
\item Options can be used in long form i.e. ls --all 
\item Options can be used in short form i.e ls -a
\end{itemize}
\end{frame}

\section{autocomplete}
\begin{frame}{Section 3: Getting help}
\begin{itemize}
\item Command line help is available as 'man' pages
\item This is short for manual
\item They can be quite detailed
\item Most commands can be used with the switch '--help'
\item As a beginner '--help' is probably easier
\end{itemize}
\end{frame}

\section{Exercises}
\begin{frame}{Section 3: Exercises}
\begin{itemize}
\item In the notes go to Section 3: Anatomy of a command
\item Read the notes for Section 3 
\item Attempt exercises 5 and 6
\item Raise your hand if you are stuck
\item We can demonstrate or explain an exercise
\end{itemize}
\end{frame}

